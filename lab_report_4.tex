%%%%%%%%%%%%%%%%%%%%%%%%%%%%%%%%%%%%%%%%%
% University/School Laboratory Report
% LaTeX Template
% Version 3.1 (25/3/14)
%
% This template has been downloaded from:
% http://www.LaTeXTemplates.com
%
% Original author:
% Linux and Unix Users Group at Virginia Tech Wiki 
% (https://vtluug.org/wiki/Example_LaTeX_chem_lab_report)
%
% License:
% CC BY-NC-SA 3.0 (http://creativecommons.org/licenses/by-nc-sa/3.0/)
%
%%%%%%%%%%%%%%%%%%%%%%%%%%%%%%%%%%%%%%%%%

%----------------------------------------------------------------------------------------
%	PACKAGES AND DOCUMENT CONFIGURATIONS
%----------------------------------------------------------------------------------------

\documentclass{article}

%\usepackage[version=3]{mhchem} % Package for chemical equation typesetting
\usepackage{siunitx} % Provides the \SI{}{} and \si{} command for typesetting SI units
\newcommand{\gc}{\degreeCelsius}
\usepackage{graphicx} % Required for the inclusion of images
\usepackage{natbib} % Required to change bibliography style to APA
\usepackage{amsmath} % Required for some math elements 
\usepackage[utf8]{inputenc} % Required for special characters processing
%
\usepackage{fullpage} % changes the margin
\setlength\parindent{0pt} % Removes all indentation from paragraphs

\renewcommand{\labelenumi}{\alph{enumi}.} % Make numbering in the enumerate environment by letter rather than number (e.g. section 6)

\usepackage{times} % Uncomment to use the Times New Roman font

%----------------------------------------------------------------------------------------
%	DOCUMENT INFORMATION
%----------------------------------------------------------------------------------------

\title{\textbf{Atmospheric Sounding: the Skew-T -- Log p Diagram} \\ Meteorology
  \& Climatology} % Title

\author{Environmental Science Degree} % Author name

\date{\today} % Date for the report

\begin{document}

\maketitle % Insert the title, author and date

\begin{center}
\begin{tabular}{l r}
Date Performed: & \underline{\hspace{3cm}}  \\ % Date the experiment was performed
Partners: & \underline{\hspace{6cm}} \\ % Partner names
& \underline{\hspace{6cm}} \\
& \underline{\hspace{6cm}} \\
Instructor: & Prof.~F.~Pérez-Bernal % Instructor/supervisor
\end{tabular}
\end{center}

% If you wish to include an abstract, uncomment the lines below
\begin{abstract}
  In this lab session students learn what is an atmospheric sounding
  and how to plot the dewpoint and temperature data from the sounding
  in a \textit{Skew-T -- Log p}. From the diagram they will learn how to
  graphically calculate relevant atmospheric parameters as well as how
  to predict the outcome of basic atmospheric configurations using the
  encoded information.
\end{abstract}

%----------------------------------------------------------------------------------------
%	SECTION 1
%----------------------------------------------------------------------------------------
\section{Introduction}
The \textit{Skew-T -- Log p} diagram is probably the most commonly used thermodynamic
diagram in weather analysis and forecasting. A large number of meteorological
variables, indices, and atmospheric conditions can be found directly
or through simple analytical procedures. Typically, the environmental
temperature, dewpoint temperature, wind speed and wind direction at
various pressure levels are plotted on the diagram \cite{awstr79006}. This plot is
commonly called a ``sounding'' \cite{FM-3}.

Let's assume that the following data have been obtained from a sounding
\begin{center}
\begin{tabular}{|cccc|}
\hline
Pressure (hPa) &  Height (m) & Temperature ($^\circ$C) &Dewpoint ($^\circ$C)  \\
\hline
 1000.0       &  30         & 10.0         &  5.0  \\
  900.0       & 930         &  4.0         &  2.0  \\% alpha = 7.8 K/km EC
  850.0       &1400         &  0.0         & -3.0  \\% alpha = 8.5 K/km EC
  750.0       &2312         &-10.0         &-12.0  \\% alpha = 7.7 K/km EC
  650.0       &3392         &-12.0         &-19.0  \\% alpha = 3.7 K/km E
  500.0       &5460         &-20.0         &-37.0  \\% alpha = 3.9 K/km E
  400.0       &7070         &-34.0         &-38.0  \\% alpha = 8.7 K/km EC
  200.0       &11600        &-70.0         &-75.0  \\   
\hline
\end{tabular}
\end{center}

According to this data, and with the help of the \textit{Skew-T -- Log p}
diagram presentation that can
be found in Moodle, students are requested to accomplish several tasks.

\section{Objectives}

\begin{enumerate}
  \item Plot temperature and dewpoint curves
  \item Compute atmospheric parameters from the sounding data.
  \item Determine atmospheric stability.
  \item Find levels of interest.
  \item Find positive and negative potential energy areas.
  \item Solve basic atmospheric situations.
\end{enumerate}

\section{Results}

\subsection{Plot Temperature and Dewpoint Curves in the Diagram}

On each given level isobar, plot the temperature to the
nearest tenths of degree Celsius. Each temperature point is represented by a small dot
located at the proper temperature -- pressure intersection. A small
circle, should be drawn around each dot plotted. 

After plotting all levels for which data was given in the previous table, connect each
point to the next with a solid line. Use a ruler
when drawing the connecting lines between the plotted points.


Repeat the previous procedure for the dewpoint data, using a different color to
distinguish both curves.

\subsection{Compute Atmospheric Parameters}

Compute on the diagram following the procedure explained in the lab
the following atmospheric parameters:

\begin{itemize}

\item Mixing ratio ($m$) in isobars $1000$, $750$, and $\SI{500}{hPa}$.

\begin{tabular}{|c|c|c|c|}
\hline
Mixing Ratio&\SI{1000}{hPa} &\SI{750}{hPa} &\SI{500}{hPa} \\
\hline
$m$ &~\hspace{2.25cm}~&~\hspace{2.25cm}~&~\hspace{2.25cm}~\\
\hline
\end{tabular}


\item Saturation mixing ratio ($M$) in isobars $1000$, $750$, and $\SI{500}{hPa}$.

\begin{tabular}{|c|c|c|c|}
\hline
Sat.~Mixing Ratio&\SI{1000}{hPa} &\SI{750}{hPa} &\SI{500}{hPa} \\
\hline
$M$ &~\hspace{2.25cm}~&~\hspace{2.25cm}~&~\hspace{2.25cm}~\\
\hline
\end{tabular}


\item Relative humidity ($h$) in isobars $1000$, $750$, and $\SI{500}{hPa}$.

\begin{tabular}{|c|c|c|c|}
\hline
Rel.~hum.&\SI{1000}{hPa} &\SI{750}{hPa} &\SI{500}{hPa} \\
\hline
$h$ &~\hspace{2.25cm}~&~\hspace{2.25cm}~&~\hspace{2.25cm}~\\
\hline
\end{tabular}

\item Vapor tension ($e$) in isobars $1000$, $750$, and $\SI{500}{hPa}$.

\begin{tabular}{|c|c|c|c|}
\hline
Vapor Pressure&\SI{1000}{hPa} &\SI{750}{hPa} &\SI{500}{hPa} \\
\hline
$e$ &~\hspace{2.25cm}~&~\hspace{2.25cm}~&~\hspace{2.25cm}~\\
\hline
\end{tabular}

\end{itemize}

\subsection{Atmospheric Stability}

According to the plotted data, find out the stability of the
atmosphere in the following intervals.

\vspace{0.25cm}
\begin{tabular}{c|c|c|c|c|c|}
 &\SI{1000}{hPa}--\SI{900}{hPa} &\SI{900}{hPa}--\SI{850}{hPa} &\SI{850}{hPa}--\SI{750}{hPa} &\SI{750}{hPa}--\SI{650}{hPa} &\SI{650}{hPa}--\SI{500}{hPa} \\
\hline
AU/AS/CU\footnotemark &~\hspace{1cm}~&~\hspace{1cm}~&~\hspace{1cm}~&~\hspace{1cm}~&~\hspace{1cm}~\\
\hline
\end{tabular}
\footnotetext{AU: absolute unstable, AS: absolutely stable, CU:
  conditionally unstable.}

\subsection{Levels of Interest}

Mark in your diagram the existing levels of interest and fill the following table.

\vspace{0.25cm}
\begin{tabular}{c|c|c|c|c|c|}
 & LCL & CCL  & LFC & ELm & ELc \\
\hline
Level Height (m) &~\hspace{2cm}~&~\hspace{2cm}~&~\hspace{2cm}~&~\hspace{2cm}~&~\hspace{2cm}~\\
\hline
\end{tabular}

\subsection{Positive and Negative Areas}

Mark in your diagram the positive and negative potential energy areas.

\subsection{Situation 1: Mechanical Lifting}
In an atmospheric state given by the plotted sounding, assume that an
air parcel driven by the wind finds a mountain in its way and is forced to ascend from the \SI{1000}{hPa} to the
\SI{750}{hPa} isobar to return afterwards to the initial sea level.

Find using the diagram
\begin{enumerate}
\item The air temperature at the mountain top and once the air returns
  to sea level.
\item Relative humidity in the top of the mountain and downwards, back
  in sea level.
\item The mixing ratio increment.
\end{enumerate}

\vspace{0.25cm}
\begin{tabular}{c|c|c|c|c|c|}
 & $T_{top}$ & $T_{bottom}$  & $h_{top}$ & $h_{bottom}$  & $\Delta m$ \\
\hline
Solution &~\hspace{2cm}~&~\hspace{2cm}~&~\hspace{2cm}~&~\hspace{2cm}~&~\hspace{2cm}~\\
\hline
\end{tabular}
\subsection{Situation 2: Convective Lifting}
In an atmospheric state given by the plotted sounding, an air parcel
at sea level is heated until it reaches a temperature $T_0 =
\SI{20}{\gc}$ and at this point starts a convective lifting.

Find using the diagram
\begin{enumerate}
\item The relative humidity and dewpoint of the air when it starts to
  lift.
\item Does this parcel of air attain its convective equilibrium
  height? If so, what is this height?
\item Are clouds formed due to the convective lift of this air parcel?
  If so, at what height will be the basis and the top of the clouds? 
\end{enumerate}

\vspace{0.25cm}
\begin{tabular}{c|c|c|c|c|c|}
 & $h_{0}$ & $\tau_{0}$  & $z_{conv}$ & $z_{basis}$  & $z_{top}$ \\
\hline
Solution &~\hspace{2cm}~&~\hspace{2cm}~&~\hspace{2cm}~&~\hspace{2cm}~&~\hspace{2cm}~\\
\hline
\end{tabular}

%----------------------------------------------------------------------------------------

\bibliographystyle{apalike}

\bibliography{sample}

%----------------------------------------------------------------------------------------


\end{document}